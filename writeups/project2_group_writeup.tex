\documentclass[10pt,english,a4paper]{article}
\usepackage{fullpage}
\usepackage[T1]{fontenc}
\usepackage{hyperref}
\usepackage{longtable}

\title{CS444 Project 2: Group 4 Write-up}
\author{
	Nelson, Nicholas\\
	\texttt{nelsonni@onid.oregonstate.edu}
	\and
	McNeil, Jonathan\\
	\texttt{mcneilj@onid.oregonstate.edu}
	\and
	Wang, Kui\\
	\texttt{wangku@onid.oregonstate.edu}
}

\begin{document}
\maketitle
	
\section{Design}
This assignment required additional research to determine the exact characteristics of a Shortest Seek Time First (SSTF) I/O Scheduler, and to develop the accompanying design and algorithm.

SSTF attempts to reduce the number of seeks required to complete a series of read/write operations on a mounted volume (hard drive, floppy drive, flash drive), thus reducing the latency and potentially increasing the throughput of the device. It accomplishes this by prioritizing blocks that are closest to the current disk head and servicing them in order of shortest distance (number of seeks) required to reach a subsequent request block.

This algorithm has a fundamentally design flaw that makes it a bad production choice on any system other than the most untaxed, underutilized systems. When sufficient loads of requests are submitted at similar times, the SSTF I/O Scheduler will prioritize based on seek distances at the detriment to request blocks that might be far away from the current disk head. This can lead to these request blocks never being serviced, and thus being "starved".

Our design focuses on the ordering of request blocks and does not provide any additional facility for merging requests; the elevator.c code is still present and providing merging and sorting facilities however. We accomplish all insertion, removal, and sorting through the use of current, previous, and next pointers to keep track of all necessary locations within the queue. As items are added, removed, or sorted, each of these pointers are updated accordingly and can thus be relied upon to contain information relating the current state of the I/O requests on the specific mounted volume.

As in our previous assignment, Project 1, we maintained our team choice of using Git and GitHub as our version control platform. We added the necessary \texttt{block/sstf\_iosched.c} file with our SSTF I/O Scheduler code and updated the \texttt{block/Kconfig.iosched} and \texttt{block/Makefile} files with the new option accordingly. Our initial attempt at building with this setup resulted in the following error however:\\\\
\texttt{make[1]: *** No rule to make target `block/sstf-iosched.o', needed by\\ `block/built-in.o'.  Stop.}\\
\texttt{make: *** [block] Error 2}\\

This issue was rooted in a naming issue that had the \texttt{block/Makefile} refering to \texttt{sstf-iosched.o} and the actual SSTF I/O scheduler file being named \texttt{sstf\_iosched.o} instead. After resolving this conflict, we were able to use the \texttt{make menuconfig}, \texttt{make}, \texttt{make modules}, and \texttt{make modules\_install} commands to compile the modified Linux Kernel. We installed it into the test hardware system using the \texttt{make install} command.
		
\section{Version Control Log}

The following version control log details all changes that were made to add the Shortest-Seek-Time-First I/O Scheduler to the Linux Kernel 3.0.4 source tree, and allow for the selection of this scheduler on mounted volumes:\\

%% This file was generated by the script latex-git-log
%% Base git commit URL: https://github.com/nelsonni/cs444-group4/commit
\newcommand{\longtableendfoot}{continues on next page}

\begin{tabular}{lp{12cm}}
  \label{tabular:legend:git-log}
  \textbf{acronym} & \textbf{meaning} \\
  V & \texttt{version} \\
  tag & \texttt{git tag} \\
  MF & Number of \texttt{modified files}. \\
  AL & Number of \texttt{added lines}. \\
  DL & Number of \texttt{deleted lines}. \\
\end{tabular}

\bigskip

\noindent
\begin{tabular}{|rllp{7.5cm}rrr|}
\hline \multicolumn{1}{|c}{\textbf{V}} & \multicolumn{1}{c}{\textbf{tag}}
& \multicolumn{1}{c}{\textbf{date}}
& \multicolumn{1}{c}{\textbf{commit message}} & \multicolumn{1}{c}{\textbf{MF}}
& \multicolumn{1}{c}{\textbf{AL}} & \multicolumn{1}{c|}{\textbf{DL}} \\ \hline

\hline 1 &  & 2014-10-09 & \href{https://github.com/nelsonni/cs444-group4/commit/d85d1e326a1dcff2678f3c9804c535df9a7f0db7}{base code, as pulled from os-class.engr.oregonstate.edu} & 36778 & 14644499 & 0 \\
\hline 2 &  & 2014-10-27 & \href{https://github.com/nelsonni/cs444-group4/commit/b114dc461867e0fb5cea8c01c5f05dfebecc1be2}{project directories clean-up, all projects now reside in their own projectX folder} & 36776 & 14644304 & 0 \\
\hline 3 &  & 2014-10-27 & \href{https://github.com/nelsonni/cs444-group4/commit/0bb5ccfc4b5b102ba8c6a3f05467579f0a67b3a6}{additional clean-up from project folders} & 36776 & 0 & 14644304 \\
\hline 4 &  & 2014-10-27 & \href{https://github.com/nelsonni/cs444-group4/commit/af58484cdeb03c7c7e21449bec642058e6b19b82}{baseline for project2} & 37576 & 14652316 & 0 \\
\hline 5 &  & 2014-10-27 & \href{https://github.com/nelsonni/cs444-group4/commit/39e794e6a06e73b6a2d85bdb04959abcccf94b62}{shortest-seek-time-first IO scheduler added} & 1 & 225 & 0 \\
\hline 6 &  & 2014-10-27 & \href{https://github.com/nelsonni/cs444-group4/commit/8b01e645062d5a732141e3424968d7e31a8f95f8}{added rules for SSTF IO scheduler to Kconfig and Make for builds} & 2 & 9 & 0 \\
\hline 7 &  & 2014-10-27 & \href{https://github.com/nelsonni/cs444-group4/commit/9ab2238a0a52b6679bf4f06ab598135e8c37893b}{write-up latex documents now live in /writeups} & 1 & 64 & 0 \\
\hline 8 &  & 2014-10-27 & \href{https://github.com/nelsonni/cs444-group4/commit/78db90ba9f2fa0c4e15b24608ab9a1da653ba4ab}{fixed output messages to be consistent and useful} & 1 & 16 & 20 \\
\hline 9 &  & 2014-10-27 & \href{https://github.com/nelsonni/cs444-group4/commit/f54bebb917d24b2d719438cb55fc1c6a9a21c719}{initial draft of Project 2 Group Write-up document} & 1 & 70 & 0 \\
\hline 10 &  & 2014-10-27 & \href{https://github.com/nelsonni/cs444-group4/commit/af2e0bbca66e78f4c5061ef8b37298ebd627806d}{resolved build errors resulting from filename differences} & 3 & 222 & 222 \\
\hline 11 &  & 2014-10-27 & \href{https://github.com/nelsonni/cs444-group4/commit/f2aace4adcd788030ee0aced10f0a141c3a55b99}{additional information about the Project 2 build errors provided, including solution, in the LaTeX document} & 1 & 1 & 1 \\
\hline
\end{tabular}

\end{document}