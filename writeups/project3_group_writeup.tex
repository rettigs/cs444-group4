\documentclass[10pt,english,a4paper]{article}
\usepackage{fullpage}
\usepackage[T1]{fontenc}
\usepackage{hyperref}
\usepackage{longtable}
\usepackage{amsmath}

\title{CS444 Project 3: Group 4 Write-up}
\author{
	Nelson, Nicholas\\
	\texttt{nelsonni@onid.oregonstate.edu}
	\and
	McNeil, Jonathan\\
	\texttt{mcneilj@onid.oregonstate.edu}
	\and
	Wang, Kui\\
	\texttt{wangku@onid.oregonstate.edu}
}

\begin{document}
\maketitle
	
\section{Design}
Based upon the literature in Chapter 17. Devices and Modules of the textbook \textit{Linux Kernel Development, 3rd Edition}, and combining the baseline of the \texttt{sbull.c} sample device driver available from many Linux Kernel source trees, we created a block device driver (imaginatively named RAM\_DISK) that allocates memory from available RAM and provides it back to the calling routine as a block device that can use the CryptAPI found in the BLK\_DEV\_CRYPTOLOOP module to encrypt and decrypt content as needed for read and write requests.

The device driver contains functions to allow device initialization, allocation, write into, and read from functionality, including optional queue structures for processing incoming requests. The request handling functionality also maintains the ability to do block request clustering and fulfillment in cases of sequential memory locations. The request handling provides requests modes that do simple request management (RM\_SIMPLE), FIFO queue management (RM\_NOQUEUE), and complete clustering and efficiency management (RM\_FULL) options.

Using the AES-128 encryption algorithm, available through the CryptAPI, the device encrypts during initialization and after the completion of each write/read request. The device driver also decrypts at the beginning of any read/write request.

The device driver operates in a simple format of creating and allocating space on initial request for a RAM\_DISK device, immediate encryption using the encryption key and the CryptAPI module calls, and decrypt and re-encrypt calls for each read/write request that comes into the queue and is serviced.

As in our previous assignments, we maintained our team choice of using Git and GitHub as our version control platform. We added the necessary \texttt{drivers/block/osurd.c} file with our RAM\_DISK device driver code and updated the \texttt{drivers/block/Kconfig} and \texttt{drivers/block/Makefile} files with the new option accordingly. Our \texttt{drivers/block/Kconfig} now contains the following additional option:\\\\
\texttt{config RAM\_DISK\\
\indent tristate ``CS444-Group 4 RAM Disk''\\
\indent default m\\
\indent depends on BLK\_DEV\_CRYPTOLOOP\\
\indent ---help---\\
\indent\indent RAM Disk device driver which allocates a chunk of memory and\\  
\indent\indent presents it as a block device. Also includes encryption from\\
\indent\indent the CryptoAPI to allow the block device to encrypt/decrypt data\\
\indent \indent when it is written and read.}\\

The configuration of this option allowed us to formalize the dependency between our RAM\_DISK module and the BLK\_DEV\_CRYPTOLOOP module so that the \texttt{modprobe} command would recognize and load the dependent modules in order for our block device to properly utilize the CryptAPI.

With the device driver added and enabled, we followed the standard steps for compiling, building, and linking together the Linux Kernel, as illustrated below:

\begin{enumerate}
	\item Set the kernel configuration options:\\\\
		\texttt{make menuconfig}\\\\
	For our RAM Disk device driver module to work, we had to enable 
	\texttt{Device Drivers} ${}\rightarrow{}$ \texttt{Block Devices} ${}
	\rightarrow{}$ \texttt{CS444-Group 4 RAM Disk} as a module (\texttt
	{M}).
		
	\item Compile the main kernel:\\\\
		\texttt{make -j8}
	\item Compile the kernel modules:\\\\
		\texttt{make modules}
	\item Install the kernel modules:\\\\
		\texttt{make modules\_install}
	\item Install the new kernel on the system:\\\\
		\texttt{make install}
\end{enumerate}

Using the source tree from Project 2, we attempted to get the kernel built and loaded with the new RAM Disk device driver module. The first time we ran through our compile and build process, though, we received several errors upon running the \texttt{make install} command:\\\\

\texttt{sh /usr/data/KERNEL/linux-stable/arch/x86/boot/install.sh 3.0.4 arch/x86/boot/bzImage \
\indent                System.map "/boot"\\
\indent ERROR: modinfo: could not find module autofs4\\
\indent ERROR: modinfo: could not find module fcoe\\
\indent ERROR: modinfo: could not find module 8021q\\
\indent ERROR: modinfo: could not find module libfcoe\\
\indent ERROR: modinfo: could not find module garp\\
\indent ERROR: modinfo: could not find module libfc\\
\indent ERROR: modinfo: could not find module stp\\
\indent ERROR: modinfo: could not find module llc\\
\indent ERROR: modinfo: could not find module scsi\_transport\_fc\\
\indent ERROR: modinfo: could not find module scsi\_tgt\\
\indent ERROR: modinfo: could not find module sunrpc\\
\indent ERROR: modinfo: could not find module ip6t\_REJECT\\
\indent ERROR: modinfo: could not find module nf\_conntrack\_ipv6\\
\indent ERROR: modinfo: could not find module xt\_state\\
\indent ERROR: modinfo: could not find module nf\_conntrack\\
\indent ERROR: modinfo: could not find module ip6table\_filter\\
\indent ERROR: modinfo: could not find module ip6\_tables\\
\indent ERROR: modinfo: could not find module ipv6\\
\indent ERROR: modinfo: could not find module xfs\\
\indent ERROR: modinfo: could not find module exportfs\\
\indent ERROR: modinfo: could not find module dm\_mirror\\
\indent ERROR: modinfo: could not find module dm\_region\_hash\\
\indent ERROR: modinfo: could not find module dm\_log\\
\indent ERROR: modinfo: could not find module raid456\\
\indent ERROR: modinfo: could not find module async\_raid6\_recov\\
\indent ERROR: modinfo: could not find module async\_pq\\
\indent ERROR: modinfo: could not find module raid6\_pq\\
\indent ERROR: modinfo: could not find module async\_xor\\
\indent ERROR: modinfo: could not find module xor\\
\indent ERROR: modinfo: could not find module async\_memcpy\\
\indent ERROR: modinfo: could not find module async\_tx\\
\indent ERROR: modinfo: could not find module ppdev\\
\indent ERROR: modinfo: could not find module parport\_pc\\
\indent ERROR: modinfo: could not find module parport\\
\indent ERROR: modinfo: could not find module sg\\
\indent ERROR: modinfo: could not find module i2c\_piix4\\
\indent ERROR: modinfo: could not find module i2c\_core\\
\indent ERROR: modinfo: could not find module snd\_intel8x0\\
\indent ERROR: modinfo: could not find module snd\_ac97\_codec\\
\indent ERROR: modinfo: could not find module ac97\_bus\\
\indent ERROR: modinfo: could not find module snd\_seq\\
\indent ERROR: modinfo: could not find module snd\_seq\_device\\
\indent ERROR: modinfo: could not find module snd\_pcm\\
\indent ERROR: modinfo: could not find module snd\_timer\\
\indent ERROR: modinfo: could not find module snd\\
\indent ERROR: modinfo: could not find module soundcore\\
\indent ERROR: modinfo: could not find module snd\_page\_alloc\\
\indent ERROR: modinfo: could not find module e1000\\
\indent ERROR: modinfo: could not find module ext3\\
\indent ERROR: modinfo: could not find module jbd\\
\indent ERROR: modinfo: could not find module mbcache\\
\indent ERROR: modinfo: could not find module raid1\\
\indent ERROR: modinfo: could not find module sd\_mod\\
\indent ERROR: modinfo: could not find module crc\_t10dif\\
\indent ERROR: modinfo: could not find module sr\_mod\\
\indent ERROR: modinfo: could not find module cdrom\\
\indent ERROR: modinfo: could not find module ahci\\
\indent ERROR: modinfo: could not find module pata\_acpi\\
\indent ERROR: modinfo: could not find module ata\_generic\\
\indent ERROR: modinfo: could not find module ata\_piix\\
\indent ERROR: modinfo: could not find module dm\_mod}\\

After several attempts to resolve this issue via removing our RAM Disk module from the source tree, using the \texttt{depmod} command to attempt to fix module config files, and the \texttt{defconfig} command to update any broken module references, we finally determined that the problem was that some modules were already compiled and in the source tree when we obtained it from the \texttt{os-class.engr.oregonstate.edu} server, and that a subsequent \texttt{make clean} command had wiped those out. We re-pulled a new copy from the server and applied our RAM Disk device driver patches to the source tree in order to fix the build process.
		
\section{Version Control Log}

The following version control log details all changes that were made to add the RAM Disk device driver, with CryptAPI enabled, to the Linux Kernel 3.0.4 source tree, and allow for the loading and unloading of this module during runtime:\\

%% This file was generated by the script latex-git-log
%% Base git commit URL: https://github.com/nelsonni/cs444-group4/commit
\newcommand{\longtableendfoot}{continues on next page}

\begin{tabular}{lp{12cm}}
  \label{tabular:legend:git-log}
  \textbf{acronym} & \textbf{meaning} \\
  V & \texttt{version} \\
  MF & Number of \texttt{modified files}. \\
  AL & Number of \texttt{added lines}. \\
  DL & Number of \texttt{deleted lines}. \\
\end{tabular}

\bigskip

\noindent
\begin{tabular}{|rlp{7.5cm}rrr|}
\hline \multicolumn{1}{|c}{\textbf{V}}
& \multicolumn{1}{c}{\textbf{date}}
& \multicolumn{1}{c}{\textbf{commit message}} & \multicolumn{1}{c}{\textbf{MF}}
& \multicolumn{1}{c}{\textbf{AL}} & \multicolumn{1}{c|}{\textbf{DL}} \\ \hline

\hline 1 & 2014-10-09 & \href{https://github.com/nelsonni/cs444-group4/commit/d85d1e326a1dcff2678f3c9804c535df9a7f0db7}{base code, as pulled from os-class.engr.oregonstate.edu} & 36778 & 14644499 & 0 \\
\hline 2 & 2014-10-27 & \href{https://github.com/nelsonni/cs444-group4/commit/b114dc461867e0fb5cea8c01c5f05dfebecc1be2}{project directories clean-up, all projects now reside in their own projectX folder} & 36776 & 14644304 & 0 \\
\hline 3 & 2014-10-27 & \href{https://github.com/nelsonni/cs444-group4/commit/0bb5ccfc4b5b102ba8c6a3f05467579f0a67b3a6}{additional clean-up from project folders} & 36776 & 0 & 14644304 \\
\hline 4 & 2014-10-27 & \href{https://github.com/nelsonni/cs444-group4/commit/af58484cdeb03c7c7e21449bec642058e6b19b82}{baseline for project2} & 37576 & 14652316 & 0 \\
\hline 5 & 2014-10-27 & \href{https://github.com/nelsonni/cs444-group4/commit/39e794e6a06e73b6a2d85bdb04959abcccf94b62}{shortest-seek-time-first IO scheduler added} & 1 & 225 & 0 \\
\hline 6 & 2014-10-27 & \href{https://github.com/nelsonni/cs444-group4/commit/8b01e645062d5a732141e3424968d7e31a8f95f8}{added rules for SSTF IO scheduler to Kconfig and Make for builds} & 2 & 9 & 0 \\
\hline 7 & 2014-10-27 & \href{https://github.com/nelsonni/cs444-group4/commit/9ab2238a0a52b6679bf4f06ab598135e8c37893b}{write-up latex documents now live in /writeups} & 1 & 64 & 0 \\
\hline 8 & 2014-10-27 & \href{https://github.com/nelsonni/cs444-group4/commit/78db90ba9f2fa0c4e15b24608ab9a1da653ba4ab}{fixed output messages to be consistent and useful} & 1 & 16 & 20 \\
\hline 9 & 2014-10-27 & \href{https://github.com/nelsonni/cs444-group4/commit/f54bebb917d24b2d719438cb55fc1c6a9a21c719}{initial draft of Project 2 Group Write-up document} & 1 & 70 & 0 \\
\hline 10 & 2014-10-27 & \href{https://github.com/nelsonni/cs444-group4/commit/af2e0bbca66e78f4c5061ef8b37298ebd627806d}{resolved build errors resulting from filename differences} & 3 & 222 & 222 \\
\hline 11 & 2014-10-27 & \href{https://github.com/nelsonni/cs444-group4/commit/f2aace4adcd788030ee0aced10f0a141c3a55b99}{additional information about the Project 2 build errors provided, including solution, in the LaTeX document} & 1 & 1 & 1 \\
\hline 12 & 2014-10-06 & \href{https://github.com/nelsonni/cs444-group4/commit/40b2feb8e6a8f4ba3a9e654dcc7d6da7aff70867}{finalized by removing all dev files, makefile updated accordingly} & 3 & 87 & 429 \\
\hline 13 & 2014-10-09 & \href{https://github.com/nelsonni/cs444-group4/commit/e295dc6b3888f29e76f1c994cba3933223d0e48c}{created initial project3 folder and put osurd.c file inside} & 1 & 455 & 0 \\
\hline 14 & 2014-10-09 & \href{https://github.com/nelsonni/cs444-group4/commit/4ff66e6417755868fe4daa950c8ed490af9653b8}{setup project3 folder more appropriately this time} & 37577 & 30942 & 0 \\
\hline 15 & 2014-10-09 & \href{https://github.com/nelsonni/cs444-group4/commit/f70ffd11b3fa45299bbbac1f9311156e085b6500}{updates to enable the RAM\_DISK option} & 3 & 12 & 1 \\
\hline
\end{tabular}

\end{document}