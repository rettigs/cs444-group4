\documentclass[10pt,english,a4paper]{article}
\usepackage{fullpage}
\usepackage[T1]{fontenc}
\usepackage{hyperref}
\usepackage{longtable}
\usepackage{amsmath}

\title{CS444 Project 4: Group 4 Write-up}
\author{
	Nelson, Nicholas\\
	\texttt{nelsonni@onid.oregonstate.edu}
	\and
	McNeil, Jonathan\\
	\texttt{mcneilj@onid.oregonstate.edu}
	\and
	Wang, Kui\\
	\texttt{wangku@onid.oregonstate.edu}
}

\begin{document}
\maketitle
	
\section{Design}
Based upon the literature in Chapter 12. Memory Management of the textbook \textit{Linux Kernel Development, 3rd Edition}, Linux Kernel developer blogs and websites, and the descriptive information in comments within the \texttt{/mm/slob.c} file, we updated the \texttt{slob.c} file to use the best-fit algorithm as opposed to the first-fit algorithm for SLAB block allocation.

To implement the SLAB best-fit algorithm, we had to first understand the first-fit algorithm that was currently in-use for SLOB allocations. The first-fit implementation 

The device driver contains functions to allow device initialization, allocation, write into, and read from functionality, including optional queue structures for processing incoming requests. The request handling functionality also maintains the ability to do block request clustering and fulfillment in cases of sequential memory locations. The request handling provides requests modes that do simple request management (RM\_SIMPLE), FIFO queue management (RM\_NOQUEUE), and complete clustering and efficiency management (RM\_FULL) options.

Using the AES-128 encryption algorithm, available through the CryptAPI, the device encrypts during initialization and after the completion of each write/read request. The device driver also decrypts at the beginning of any read/write request.

The device driver operates in a simple format of creating and allocating space on initial request for a RAM\_DISK device, immediate encryption using the encryption key and the CryptAPI module calls, and decrypt and re-encrypt calls for each read/write request that comes into the queue and is serviced.

As in our previous assignments, we maintained our team choice of using Git and GitHub as our version control platform. We added the necessary \texttt{drivers/block/osurd.c} file with our RAM\_DISK device driver code and updated the \texttt{drivers/block/Kconfig} and \texttt{drivers/block/Makefile} files with the new option accordingly. Our \texttt{drivers/block/Kconfig} now contains the following additional option:\\\\
\texttt{config RAM\_DISK\\
\indent tristate ``CS444-Group 4 RAM Disk''\\
\indent default m\\
\indent depends on BLK\_DEV\_CRYPTOLOOP\\
\indent ---help---\\
\indent\indent RAM Disk device driver which allocates a chunk of memory and\\  
\indent\indent presents it as a block device. Also includes encryption from\\
\indent\indent the CryptoAPI to allow the block device to encrypt/decrypt data\\
\indent \indent when it is written and read.}\\

The configuration of this option allowed us to formalize the dependency between our RAM\_DISK module and the BLK\_DEV\_CRYPTOLOOP module so that the \texttt{modprobe} command would recognize and load the dependent modules in order for our block device to properly utilize the CryptAPI.

With the device driver added and enabled, we followed the standard steps for compiling, building, and linking together the Linux Kernel, as illustrated below:

\begin{enumerate}
	\item Set the kernel configuration options:\\\\
		\texttt{make menuconfig}\\\\
	The options for setting the SLAB allocator are available in \texttt{General setup} ${}\rightarrow{}$ \texttt{Choose SLAB allocator}. However, the SLOB option has been hidden and must be enabled by enabling the \texttt{General setup} ${}\rightarrow{}$ \texttt{Configure standard kernel features (expert users)} option. After that option is set, return to \texttt{Choose SLAB allocator} and select \texttt{SLOB (Simple Allocator)}.
		
	\item Compile the main kernel:\\\\
		\texttt{make -j8}
	\item Compile the kernel modules:\\\\
		\texttt{make modules}
	\item Install the kernel modules:\\\\
		\texttt{make modules\_install}
	\item Install the new kernel on the system:\\\\
		\texttt{make install}
\end{enumerate}

Using a clean Linux Kernel 3.0.4 source tree, we attempted to get the kernel built and loaded with the SLAB allocator set to use the SLOB algorithm. The first time we ran through the compile and build proces, though, we received the following errors upon running the \texttt{make install} command:\\

\texttt{[root@localhost linux]\# make -j8\\
\indent\indent CHK \indent include/linux/version.h\\
\indent\indent CHK \indent include/generated/utsrelease.h\\
\indent\indent CALL\indent scripts/checksyscalls.sh\\
\indent\indent CHK \indent include/generated/compile.h\\
\indent\indent CHK \indent mm/slob.c\\
\indent mm/slob.c: In function `slob\_page\_alloc':\\
\indent mm/slob.c:292: error: `best\_fit' undeclared (first use in this function)\\
\indent mm/slob.c:292: error: (Each undeclared identifier is reported only once\\
\indent mm/slob.c:292: error: for each function it appears in.)\\
\indent make[1]: *** [mm/slob.o] Error 1\\
\indent make: *** [mm] Error 2\\
\indent make: *** Waiting for unfinished jobs....}\\

This issue was resolved by locating and fixing the incorrectly labeled variable; \texttt{best\_fit} should have been \texttt{best\_diff}.

After several attempts to resolve this issue via removing our RAM Disk module from the source tree, using the \texttt{depmod} command to attempt to fix module config files, and the \texttt{defconfig} command to update any broken module references, we finally determined that the problem was that some modules were already compiled and in the source tree when we obtained it from the \texttt{os-class.engr.oregonstate.edu} server, and that a subsequent \texttt{make clean} command had wiped those out. We re-pulled a new copy from the server and applied our RAM Disk device driver patches to the source tree in order to fix the build process.
		
\section{Version Control Log}

The following version control log details all changes that were made to add the RAM Disk device driver, with CryptAPI enabled, to the Linux Kernel 3.0.4 source tree, and allow for the loading and unloading of this module during runtime:\\

%% This file was generated by the script latex-git-log
%% Base git commit URL: https://github.com/nelsonni/cs444-group4/commit
\newcommand{\longtableendfoot}{continues on next page}

\begin{tabular}{lp{12cm}}
  \label{tabular:legend:git-log}
  \textbf{acronym} & \textbf{meaning} \\
  V & \texttt{version} \\
  MF & Number of \texttt{modified files}. \\
  AL & Number of \texttt{added lines}. \\
  DL & Number of \texttt{deleted lines}. \\
\end{tabular}

\bigskip

\noindent
\begin{tabular}{|rlp{7.5cm}rrr|}
\hline \multicolumn{1}{|c}{\textbf{V}}
& \multicolumn{1}{c}{\textbf{date}}
& \multicolumn{1}{c}{\textbf{commit message}} & \multicolumn{1}{c}{\textbf{MF}}
& \multicolumn{1}{c}{\textbf{AL}} & \multicolumn{1}{c|}{\textbf{DL}} \\ \hline

\hline 1 & 2014-10-09 & \href{https://github.com/nelsonni/cs444-group4/commit/d85d1e326a1dcff2678f3c9804c535df9a7f0db7}{base code, as pulled from os-class.engr.oregonstate.edu} & 36778 & 14644499 & 0 \\
\hline 2 & 2014-10-27 & \href{https://github.com/nelsonni/cs444-group4/commit/b114dc461867e0fb5cea8c01c5f05dfebecc1be2}{project directories clean-up, all projects now reside in their own projectX folder} & 36776 & 14644304 & 0 \\
\hline 3 & 2014-10-27 & \href{https://github.com/nelsonni/cs444-group4/commit/0bb5ccfc4b5b102ba8c6a3f05467579f0a67b3a6}{additional clean-up from project folders} & 36776 & 0 & 14644304 \\
\hline 4 & 2014-10-27 & \href{https://github.com/nelsonni/cs444-group4/commit/af58484cdeb03c7c7e21449bec642058e6b19b82}{baseline for project2} & 37576 & 14652316 & 0 \\
\hline 5 & 2014-10-27 & \href{https://github.com/nelsonni/cs444-group4/commit/39e794e6a06e73b6a2d85bdb04959abcccf94b62}{shortest-seek-time-first IO scheduler added} & 1 & 225 & 0 \\
\hline 6 & 2014-10-27 & \href{https://github.com/nelsonni/cs444-group4/commit/8b01e645062d5a732141e3424968d7e31a8f95f8}{added rules for SSTF IO scheduler to Kconfig and Make for builds} & 2 & 9 & 0 \\
\hline 7 & 2014-10-27 & \href{https://github.com/nelsonni/cs444-group4/commit/9ab2238a0a52b6679bf4f06ab598135e8c37893b}{write-up latex documents now live in /writeups} & 1 & 64 & 0 \\
\hline 8 & 2014-10-27 & \href{https://github.com/nelsonni/cs444-group4/commit/78db90ba9f2fa0c4e15b24608ab9a1da653ba4ab}{fixed output messages to be consistent and useful} & 1 & 16 & 20 \\
\hline 9 & 2014-10-27 & \href{https://github.com/nelsonni/cs444-group4/commit/f54bebb917d24b2d719438cb55fc1c6a9a21c719}{initial draft of Project 2 Group Write-up document} & 1 & 70 & 0 \\
\hline 10 & 2014-10-27 & \href{https://github.com/nelsonni/cs444-group4/commit/af2e0bbca66e78f4c5061ef8b37298ebd627806d}{resolved build errors resulting from filename differences} & 3 & 222 & 222 \\
\hline 11 & 2014-10-27 & \href{https://github.com/nelsonni/cs444-group4/commit/f2aace4adcd788030ee0aced10f0a141c3a55b99}{additional information about the Project 2 build errors provided, including solution, in the LaTeX document} & 1 & 1 & 1 \\
\hline 12 & 2014-10-06 & \href{https://github.com/nelsonni/cs444-group4/commit/40b2feb8e6a8f4ba3a9e654dcc7d6da7aff70867}{finalized by removing all dev files, makefile updated accordingly} & 3 & 87 & 429 \\
\hline 13 & 2014-10-09 & \href{https://github.com/nelsonni/cs444-group4/commit/e295dc6b3888f29e76f1c994cba3933223d0e48c}{created initial project3 folder and put osurd.c file inside} & 1 & 455 & 0 \\
\hline 14 & 2014-10-09 & \href{https://github.com/nelsonni/cs444-group4/commit/4ff66e6417755868fe4daa950c8ed490af9653b8}{setup project3 folder more appropriately this time} & 37577 & 30942 & 0 \\
\hline 15 & 2014-10-09 & \href{https://github.com/nelsonni/cs444-group4/commit/f70ffd11b3fa45299bbbac1f9311156e085b6500}{updates to enable the RAM\_DISK option} & 3 & 12 & 1 \\
\hline
\end{tabular}

\end{document}