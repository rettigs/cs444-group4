\documentclass[10pt,english,a4paper]{article}
\usepackage{fullpage}
\usepackage[T1]{fontenc}
\usepackage{hyperref}
\usepackage{longtable}

\title{CS444 Project 1: Group 4 Write-up}
\author{
	Nelson, Nicholas\\
	\texttt{nelsonni@onid.oregonstate.edu}
	\and
	McNeil, Jonathan\\
	\texttt{mcneilj@onid.oregonstate.edu}
	\and
	Wang, Kui\\
	\texttt{wangku@onid.oregonstate.edu}
}

\begin{document}
\maketitle
	
\section{Design}
In order to begin development of our solution, we first had to setup the supplied hardware and configure it according to the requirements outlined in the Project 1 instructions. Once connected to the Internet, and using the provided account login information, we downloaded and installed the \texttt{Additional Development}, \texttt{Desktop Platform Development}, and \texttt{Development Tools} package groups. Due to the large number of dependency packages for each of these package groups, the process took 15-20 minutes to complete.
		
As a group we decided to forgo the use of Subversion for our version control system and instead utilized Git. This decision was made out of a desire for simplicity and the use of the \texttt{latex-git-log} package to enable quick inclusion of the change logs. This decision meant that once we obtained the Linux 3.0.4 Kernel source tree from the os-class server, using Subversion, we created an initial git commit and uploaded our code to a GitHub repository (\url{https://github.com/nelsonni/cs444-group4}) and began work on the assignment objectives.
	
The Linux Kernel has had Round Robin and First-In-First-Out (FIFO) process scheduling since the earliest days prior to version 2.6 (which saw the introduction and defaulting of the Completely Fair Scheduler), we assumed that the scheduler options should be available in the unmodified source tree found at the Kernel.org website (\url{https://www.kernel.org/pub/linux/kernel/v3.0/linux-3.0.4.tar.bz2}). Once we had a downloaded copy of the canonical Linux Kernel 3.0.4 version, we were able to compare against our provided kernel source tree to determine where any discrepencies were located.
	
Using the \texttt{git diff -name-only <path> <path>} command, we were able to determine that the only changes were in the \texttt{kernel/sched\_rt.c} file. And within this file, we were able to determine all changes between the two source trees using the \texttt{git diff <path> <path>} command.
		
\section{Version Control Log}

The following version control log details all changes that were made to bring the source tree into alignment with the Linux Kernel 3.0.4 source tree and allow for the use of Round Robin and First-In-First-Out (FIFO) process scheduling:\\

%% This file was generated by the script latex-git-log
%% Base git commit URL: https://github.com/nelsonni/cs444-group4/commit
\newcommand{\longtableendfoot}{continues on next page}

\begin{tabular}{lp{12cm}}
  \label{tabular:legend:git-log}
  \textbf{acronym} & \textbf{meaning} \\
  V & \texttt{version} \\
  tag & \texttt{git tag} \\
  MF & Number of \texttt{modified files}. \\
  AL & Number of \texttt{added lines}. \\
  DL & Number of \texttt{deleted lines}. \\
\end{tabular}

\bigskip

\noindent
\begin{tabular}{|rllp{7.5cm}rrr|}
\hline \multicolumn{1}{|c}{\textbf{V}} & \multicolumn{1}{c}{\textbf{tag}}
& \multicolumn{1}{c}{\textbf{date}}
& \multicolumn{1}{c}{\textbf{commit message}} & \multicolumn{1}{c}{\textbf{MF}}
& \multicolumn{1}{c}{\textbf{AL}} & \multicolumn{1}{c|}{\textbf{DL}} \\ \hline

\hline 1 &  & 2014-10-09 & \href{https://github.com/nelsonni/cs444-group4/commit/d85d1e326a1dcff2678f3c9804c535df9a7f0db7}{base code, as pulled from os-class.engr.oregonstate.edu} & 36778 & 14644499 & 0 \\
\hline 2 &  & 2014-10-09 & \href{https://github.com/nelsonni/cs444-group4/commit/0809d743782fb6eb204341ba7bc6529343e42965}{added SCHED\_RR and SCHED\_FIFO logic to sched\_rt} & 1 & 10 & 2 \\
\hline 3 &  & 2014-10-09 & \href{https://github.com/nelsonni/cs444-group4/commit/eea55c2ab4eff8faf434e1d87bd1a6f3d6ab9c05}{modified gitignore to allow .config file that Kevin D. McGrath generated for this project} & 2 & 5195 & 0 \\
\hline
\end{tabular}

\end{document}